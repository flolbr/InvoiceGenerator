%% !TEX TS-program = xelatex
% !TEX encoding = UTF-8 Unicode

%%%%%%%%%%%%%%%%%
% Jinja imports %
%%%%%%%%%%%%%%%%%

<+ import "contact.jinja.tex" as contact with context +>

<+ set info = document.info +>

<+ set client = document.data.client +>
<+ set me = document.data.me +>
<+ set nomenclature = document.data.nomenclature +>

%%%%%%%%%%%%%%%%%%
% Document Class %
%%%%%%%%%%%%%%%%%%

\documentclass[10pt,a4paper]{article}

%%%%%%%%%%%%
% Packages %
%%%%%%%%%%%%

\usepackage{fontspec}
\setmainfont{Arial}
\defaultfontfeatures{Mapping=tex-text}
\usepackage{xunicode}
\usepackage{xltxtra}
%\setmainfont{???}
\usepackage{polyglossia}
\setdefaultlanguage{french}
\usepackage{amsmath}
\usepackage{amsfonts}
\usepackage{amssymb}
\usepackage[table]{xcolor}
\usepackage[left=1.27cm,right=1.27cm,top=1.27cm,bottom=1.27cm]{geometry}
\usepackage{tabularray}

%%%%%%%%%%%%%
% Variables %
%%%%%%%%%%%%%

\definecolor{AccentColor}{HTML}{3897d4}


%%%%%%%%%%%%
% Document %
%%%%%%%%%%%%

\begin{document}
    ~\\ % new line to avoid alinea
    \begin{minipage}{0.6\textwidth}
        \begin{flushleft} \large

            <@ info.title @>

        \end{flushleft}
    \end{minipage}
    % ~
    \begin{minipage}{0.4\textwidth}
        \begin{flushright} % \large

            \begin{tblr}{
                colspec      = { c c },
                rows         = {bg = gray!10},
                row{odd}     = {bg = gray!10},
                row{1}       = {fg = white, bg = AccentColor},
                hline{1,2,Z} = {0.5pt},
                vline{1,Z}   = {0.5pt},
                stretch      = 1.5
            }
                \SetCell[c=2]{c} \textbf{<@ document.type | upper @>} &                       \\
                <@ document.type @> n°                                & <@ document.number @> \\
                Date                                                  & <@ document.date @>   \\
            \end{tblr}

        \end{flushright}
    \end{minipage}\\ [1cm]

    ~\\ % new line to avoid alinea
    \begin{minipage}{0.5\textwidth}
        \begin{flushleft} % \large

            <@ contact.block(me) @>

        \end{flushleft}
    \end{minipage}
    % ~
    \begin{minipage}{0.4\textwidth}
        \begin{flushright} % \large

            <@ contact.block(client) @>

        \end{flushright}
    \end{minipage}\\ [1cm]


    ~\\ % new line to avoid alinea
    \begin{tblr}{
        colspec      = { X[l,6] X[c] X[c] X[c] },
        row{odd}     = {bg = gray!10},
        row{1}       = {fg = white, bg = AccentColor},
        hline{1,2,Z} = {0.5pt},
        vline{1,Z}   = {0.5pt},
        stretch      = 1.5
    }
        \textbf{Désignation} & \textbf{PU TTC}  & \textbf{Qté}   & \textbf{Total TTC} \\
        <+ for item in nomenclature.items +>
        <@ item.item @>      & <@ item.price @> & <@ item.qty @> & <@ item.total @>   \\
        <+ endfor +>

    \end{tblr}

%    \,\\ % new line to avoid alinea
    \begin{flushright}

        \large{TOTAL TTC: <@ nomenclature.total @>} \\
        \textcolor{gray!80}{\small{\em{TVA non applicable, art. 293B du CGI}}}  \\

    \end{flushright}

    \,\\ % new line to avoid alinea
    \begin{tblr}{
        colspec      = { l r },
%        row{odd}     = {bg = gray!10},
%        row{1}       = {fg = white, bg = AccentColor},
%        hline{1,2,Z} = {0.5pt},
%        vline{1,Z}   = {0.5pt},
        stretch      = 1.25
    }
        Début des prestations:     & <@ info.start_date @>    \\
        Livraison des prestations: & <@ info.delivery_date @> \\
    \end{tblr}

    \vspace*{\fill}

    \paragraph{}
    \emph{
        Le paiement doit s’effectuer à la livraison de la prestation par chèque ou virement bancaire. \\
    }

    \paragraph{}
    \emph{
%        Le paiement doit s’effectuer à la livraison de la prestation par chèque ou virement bancaire. \\
        Il est convenu que le vendeur reste propriétaire des marchandises, matérielles et immatérielles, vendues tant que l’acquéreur ne lui a pas entièrement réglé le prix prévu dans le présent contrat.
        Il en résulte qu’en cas de non-paiement, le vendeur pourra exiger à tout moment la restitution desdites marchandises.
    }
\end{document}